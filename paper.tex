\documentclass{article}
\usepackage{natbib}

\title{High-Performance Data Analysis on Janus using Apache Spark}
\author{Nick Vanderweit \\
        Ning Gao \\
        Anitha Ganesha \\
        Michael Kasper}

\begin{document}
\maketitle

\section{Introduction}
Over the past decade, there has been considerable interest in high-level
strategies for evaluating algorithms in parallel over large data sets.
Google's seminal MapReduce paper \citep{dean-mapreduce} describes such a
general strategy, which had already been in use at Google at the time of
publication, for building highly-scalable parallel applications out of small
serial functions. In general, phrasing data parallelism in terms of
higher-order functions on parallel data structures has proven fruitful
for many applications. In this paper, we evaluate Spark, a library
designed to address some of the shortcomings of MapReduce, on an existing
supercomputer.

In the MapReduce model, data is broken into \emph{splits} of a given size
before processing. These are stored on a distributed filesystem as
key/value pairs. The master node then distributes splits among the remaining
workers. Each of these workers runs a \emph{Map} on its split, computing for each
key/value pair $(k_1, v_1)$ another pair $(k_2, v_2)$. Each of these output
pairs is stored on the filesystem and the master is informed of their
locations.

The master node proceeds by forwarding these intermediate pairs to the
\emph{Reduce} workers, which are responsible for combining the results
for each intermediate key.

\bibliography{paper}
\bibliographystyle{plain}

\end{document}
